\documentclass[10pt,a4paper,twoside]{article}
\usepackage{graphicx}
\usepackage{natbib}

\begin{document}

\title{Calkulate}
\author{Matthew P. Humphreys}
\maketitle

\section{Software set-up}

\subsection{Installation}\label{sx:install}

Installation instructions for Python (steps \ref{li:pystart}--\ref{li:pyend}) and MATLAB (\ref{li:pystart}--end):

\begin{enumerate}

  \item\label{li:pystart} Download and install Anaconda (https://www.anaconda.com/download/). Choose the Python 3.X version for preference (although the 2.X should work too, if you already have it)
  
  \item Open the Anaconda Prompt (Windows) or Terminal (Mac/Linux)
  
  \item Create a new Python 3.6 environment by entering the following:
\begin{verbatim}conda create -n calkenv python=3.6 numpy=1.15 scipy=1.1\end{verbatim}

  \item Activate the new environment
  
Mac/Linux:
\begin{verbatim}source activate calkenv\end{verbatim}

Windows:
\begin{verbatim}activate calkenv\end{verbatim}

  You should now see the environment's name (i.e .\texttt{calkenv}) appear in brackets at the start of each line in the Anaconda Prompt/Terminal

  \item Install the Calkulate package into the environment using pip:
\begin{verbatim}pip install calkulate\end{verbatim}

  \item\label{li:pyend} You should now be able to use Calkulate in this Python environment. If you wish to also use Calkulate in MATLAB, continue to step \ref{li:matstart} onwards
  
  \item\label{li:matstart} Still within the Anaconda Prompt/Terminal (making sure that the calkenv environment is active), run Python:
\begin{verbatim}python\end{verbatim}

  \item Find the location of this environment's Python executable by entering the following 2 lines:
\begin{verbatim}
from sys import executable
print(executable)
\end{verbatim}

  \item\label{li:pyexe} Copy the string that appears. It should look something like:
  
  ‘C:\textbackslash Users\textbackslash username\textbackslash anaconda\textbackslash Anaconda3\textbackslash envs\textbackslash calkenv\textbackslash python.exe’
  
  This string is the value for the \texttt{python\_exe} variable that goes into the MATLAB function \texttt{calk\_initpy()}
  
  \item Exit python
\begin{verbatim}exit()\end{verbatim}
 
  \item Download the MATLAB wrappers (LINK). These are a set a functions that make it easier for you to use some parts of Calkulate within MATLAB, although they are just for convenience -- it's possible to use the entire program without them.
  
  \item Move the downloaded folder to a sensible location, and add it (plus all subfolders) to your MATLAB search path
  
  \item Before you can execute the MATLAB functions you must first run the \texttt{calk\_initpy()} function at least once (per MATLAB session), with the input \texttt{calk\_initpy()} string obtained in step \ref{li:pyexe}

\end{enumerate}

\subsection{Updates}

\subsubsection{Python}

Update instructions for Python:

\begin{enumerate}

  \item Open the Anaconda Prompt
  
  \item Activate the calkenv environment (installation instructions, step 4)

  \item Upgrade the Calkulate package using pip:

\begin{verbatim}
pip install calkulate --upgrade --no-cache-dir
\end{verbatim}

\end{enumerate}

\subsubsection{MATLAB}

Update instructions for MATLAB:

\begin{enumerate}

  \item Delete your original Calkulate scripts
  
  \item Replace them in full with the new versions

\end{enumerate}


\section{Testing}

\subsection{MATLAB}

You could quickly test that Calkulate is working in MATLAB by running the following, with \texttt{python\_exe} first changed to the correct string (see Section \ref{sx:install}, step \ref{li:pyexe}):

\begin{verbatim}
calk_initpy(python_exe)
[Macid,pH,Tk,Msamp,Cacid,S,XT,KX] = calk_Dickson1981;
\end{verbatim}

This should import the simulated titration data from Table 1 of \citet{dickson1981}. A plot of the Free scale pH (\texttt{pH}) against the acid mass (\texttt{Macid}) should appear as follows:

\begin{figure}[h]
\centering
\includegraphics[width=10cm]{figures/Macid_pH_D81.png}
\caption{Simulated titration data from Table 1 of \citet{dickson1981}.}
\label{fig:dickson1981}
\end{figure}

\section{References}

\bibliographystyle{elsarticle-harv}
\bibliography{Calkulate}


\end{document}
